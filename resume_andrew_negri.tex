%%%%%%%%%%%%%%%%%%%%%%%%%%%%%%%%%%%%%%%%%
% Medium Length Professional CV
% LaTeX Template
% Version 2.0 (8/5/13)
%
% This template has been downloaded from:
% http://www.LaTeXTemplates.com
%
% Original author:
% Trey Hunner (http://www.treyhunner.com/)
%
% Important note:
% This template requires the resume.cls file to be in the same directory as the
% .tex file. The resume.cls file provides the resume style used for structuring the
% document.
%
%%%%%%%%%%%%%%%%%%%%%%%%%%%%%%%%%%%%%%%%%

%----------------------------------------------------------------------------------------
%	PACKAGES AND OTHER DOCUMENT CONFIGURATIONS
%----------------------------------------------------------------------------------------

\documentclass{resume} % Use the custom resume.cls style

\usepackage[left=0.75in,top=0.6in,right=0.75in,bottom=0.6in]{geometry} % Document margins

%\newcommand\date[1]{\textit{#1}}
%\def\dates#1{\datesaux#1\stop}
%\def\datesaux#1-#2\stop{
%  \date{#1}\thinspace--\thinspace\date{#2}}

\name{Andrew Negri} % Your name
\address{7017 S. Priest Dr. \#3111 \\ Tempe, AZ 85283} % Your address
\address{(949)~$\cdot$~878~$\cdot$~2232 \\ aenegri@asu.edu} % Your phone number and email

\begin{document}

%----------------------------------------------------------------------------------------
%	EDUCATION SECTION
%----------------------------------------------------------------------------------------

\begin{rSection}{Education}

{\bf Arizona State University} \hfill {\em Expected May 2018} \\
B.S. in Software Engineering, Minor in Business \\
Barrett Honors College \smallskip \\
Overall GPA: 3.72 \textit{Dean's List}

\end{rSection}

%----------------------------------------------------------------------------------------
%	EMPLOYMENT SECTION
%----------------------------------------------------------------------------------------

\begin{rSection}{Employment}

\begin{rSubsection}{GoDaddy}{May 2017 -- August 2017}{Software Development Intern}{Gilbert, AZ}
\item Created a plugin in Python that captured the web access data of over 600,000 customer sites for Hadoop processing in real time. Developed using CI/CD with Jenkins and deployed using Puppet.
\item Discovered over 1,000,000 customer data files that had not been deleted in the account deletion workflow and wrote a Bash script to manually remove them.
\item Built and deployed 5 proxy servers by following an extensive build process with Puppet and other command line tools.
\end{rSubsection}

\begin{rSubsection}{Arizona State University}{August 2015 -- December 2015}{ASU 101 Section Leader}{Tempe, AZ}
\item Led two sections of about 30 students each in an introductory course for the Software Engineering program at ASU for 50 minutes a week.
\item Collaborated with faculty members on each lesson plan to optimize the content for the students.
\end{rSubsection}

\end{rSection}

%----------------------------------------------------------------------------------------
%	PROJECTS SECTION
%----------------------------------------------------------------------------------------

\begin{rSection}{Projects}

\begin{rSubsection}{Real Time Event Notification System}{July 2017}{GoDaddy Intern Hackathon}{}
\item Created a REST API in Python using the Django framework that received event data and alerted relevant groups via a custom Slack bot.
\item Developed a custom Amazon Alexa skill to accurately parse natural language and post to the API.
\end{rSubsection}
\begin{rSubsection}{Drill Keeper Android App}{February 2016 -- Present}{}{}
\item Developed an application that serves as a digital drill book for use in marching bands. By entering data for a marching show, a performer can easily reference their coordinates and receive navigation instructions between each set.
\item Crafted an intuitive user interface that adheres to material design guidelines with XML using Android UI components.
\item Saved and passed data by serializing objects with JSON via the Gson library.
\item Released on the Google Play store and updated regularly. Used user feedback to find bugs and influence future development.
\end{rSubsection}

\end{rSection}

%----------------------------------------------------------------------------------------
%	TECHNICAL STRENGTHS SECTION
%----------------------------------------------------------------------------------------

\begin{rSection}{Programming Languages}

\begin{tabular}{ @{} >{\bfseries}l @{\hspace{6ex}} l }
Proficient & Java, C++, Python, Bash, JavaScript \\
Familiar & Verilog, MIPS Assembly
\end{tabular}

\end{rSection}



%----------------------------------------------------------------------------------------

\end{document}